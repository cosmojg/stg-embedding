
% This LaTeX was auto-generated from an M-file by MATLAB.
% To make changes, update the M-file and republish this document.

\documentclass{article}
\usepackage{graphicx}
\usepackage{color}

\sloppy
\definecolor{lightgray}{gray}{0.5}
\setlength{\parindent}{10pt}
\usepackage[margin=1in]{geometry}

\begin{document}

\title{Embedding STG dynamics}
\author{Srinivas Gorur-Shandilya}
\date{\today}
\maketitle

    
    
\subsection*{Contents}

\begin{itemize}
\setlength{\itemsep}{-1ex}
   \item The data
   \item Measuring periods: the classical way
   \item Measuring phases: the classical way
   \item Extracting periods using the Hilbert transform
   \item Computing phases differences using the Hilbert transform
   \item Estimating phases differences form cross correlation functions
   \item Version Info
\end{itemize}
\begin{par}
In this document, I look at how one can estimate phases of oscillating (bursting) neurons from their spike times, and how one can measure phase differences between two oscillating neurons.
\end{par} \vspace{1em}


\subsection*{The data}

\begin{par}
The data I use here is intracellular recordings of LP and PD neurons from Jess Haley. I show a short raster of this data below:
\end{par} \vspace{1em}

\includegraphics [width=\textwidth]{estimating_phases_01.pdf}
\begin{par}
Now I plot the ISI distributions of PD and LP.
\end{par} \vspace{1em}

\includegraphics [width=\textwidth]{estimating_phases_02.pdf}
\begin{par}
Note that it is extremely bimodal -- corresponding to bursting and spiking. This lets one cleanly identify bursts in the time trace by finding the time of the first and last spike of every burst. There are some weird things -- note that the "burst ISI" of the two neurons don't match up exaclty -- which is suprising, because on would expect them to be coupled.
\end{par} \vspace{1em}


\subsection*{Measuring periods: the classical way}

\begin{par}
Now, I use the timing of the first and last spikes to measure the period of the PD and LP neurons, as is done traditionally. Since every burst of a neuron has a first and a last spike, there are two indepndent ways of measuring the period. In (a), I plot histograms of the burst period as measured using the first spike (green) and the last spike (red) for the entire dataset. Note that the periods estiamted from the last spike are a lot more variable -- this is because spiking is a probabilistic process, and a close inspection of the bursts of PD reveals that sometimes, the "last" spike doesn't occur. Plotting the periods measured in these two ways reveals striking correlations (b), that once again arise from the fact that spiking is probabilistic. A similar effect, albiet much weaker, is observed for LP (c-d).
\end{par} \vspace{1em}

\includegraphics [width=\textwidth]{estimating_phases_03.pdf}


\subsection*{Measuring phases: the classical way}

\begin{par}
Now we measure "phases" using the traditional way by finding the time delay between the last spike in PD and the first spike in LP, and normalizing by the burst period.
\end{par} \vspace{1em}

\includegraphics [width=\textwidth]{estimating_phases_04.pdf}


\subsection*{Extracting periods using the Hilbert transform}

\begin{par}
In this section, I attempt to extract periods of the neurons from their spike trains, but by first extracting the true phase of the neuron using the Hilbert transform. For a discussion on extracting phases and amplitudes from time series, see "Synchronization"  by Pikovsky, Rosenblum \& Kurths (Appendix A2).
\end{par} \vspace{1em}
\begin{par}
In the following figure, I compare the rasters of PD and LP to the extracted phases of LP and PD. Note that the phase of PD is 0 at the onset of PD bursting (by construction). Note however, that the phase of LP also tends to be 0 at the onset of LP bursting -- which naturally comes out of the data, showing that the phase reconstruction captures the burst onsets well.
\end{par} \vspace{1em}

\includegraphics [width=\textwidth]{estimating_phases_05.pdf}
\begin{par}
In (a), I compare the periods extracted from the first and last spikes (green, red), and from the Hilbert transform. Note that the periods from the Hilbert transform agree well with the traditional way, but the distribution is more well-behaved and more mono-modal. In (b), I plot the true phase vs. the normalized time since phase onset. The error bars are standard deviations across all periods, and the dotted line is the assumption of linearity (as in the traditional method). Finally, in the panel on the right, I plot the amplitude-phase diagram of the time series, showing every oscilaltion. Data from PD is shown in (a-b), and data from LP is shown in (c-d).
\end{par} \vspace{1em}

\includegraphics [width=\textwidth]{estimating_phases_06.pdf}


\subsection*{Computing phases differences using the Hilbert transform}

\begin{par}
In this section, I measure phase differences from the continuous estimate of phase that the Hilbert transform gives me for each neuron. Since we know the phases of every neuron at every time point, the phase difference is simply the difference of these two time series. (a) compares the phase differences extracted from the Hilbert transform to the phase differences measured from the first and last spikes. Note that the Hilbert-transform phase differneces are inbetween the phase differences as reported by the first and last spikes, which makes sense. In (b), I compare the phase differences between the spike trains and between the raw intracellular voltage. Note that they don't exactly match up, and the distribution for the intracellular voltage is much tighter -- consistent with the idea of stochastic spike generation underyling variability in phase responses.
\end{par} \vspace{1em}

\includegraphics [width=\textwidth]{estimating_phases_07.pdf}


\subsection*{Estimating phases differences form cross correlation functions}

\begin{par}
In this section, I use the crosscorrelation function to estimate the phase difference between LP and PD. Note that both delays and periods can be inferred from a single cross correlation function, and I define the phase offset as the ratio of the delay to the period for that snippet. Note that the phase differences calcualted this way agree with the phase differences calcualted using the Hilbert transform.
\end{par} \vspace{1em}

\includegraphics [width=\textwidth]{estimating_phases_08.pdf}


\subsection*{Version Info}


        \color{lightgray} \begin{verbatim}estimating_phases
md5 hash of file that made this is:
8a31eab2abd02538ec6e3b71f7eb5e19
it should be in this commit:
95df8215e053418715ae2f104f46bb7978c0c264

This file has the following external dependencies:
repo name:  data-manager (ebfbf7a2275855c54e59f0fec585d7ffb7533a62)
repo name:  puppeteer (fbf48970249f374bee1be54fdb503c10444adb02)
repo name:  spikesort (a1878f2005924e7d119bdad6a3a7035d7024c038)
repo name:  srinivas.gs_mtools (3176ae935fe424faf60527fa110fa853c1ce123c)
This document was built in:
28.76 seconds.
\end{verbatim} \color{black}
    


\end{document}
    
